\section{Сходимость марковских цепей}

Введем обозначения:

\begin{enumerate}
\item $T_i$ --- время до посещения состояния $i$, если начать в состоянии $i$
\end{enumerate}

Случайная величина $T_i$ особенна тем, что она может принимать нечисловое значение $+\infty$, если цепь начала в состоянии $i$, но никогда в него не вернулась. 

Для подсчета математического ожидания $\E(T_i)$ мы примем соглашение, что $0\cdot\infty=0$. Другими словами, если $\P(T_i=\infty)=0$, то $\E(T_i)=\sum k\P(T_i=k)$, а если $\P(T_i=\infty)>0$, то $\E(T_i)=\infty$.

Вероятность того, что цепь из состояния $i$ когда-либо вернется в состояние $i$ мы обозначим $f_i$, то есть $f_i=1-\P(T_i=\infty)$.

Простая классификация состояний:
\begin{enumerate}
\item $f_i=1$, $\E(T_i)<\infty$. Транзиентное, переходное, transient
\item $f_i<1$, $\E(T_i)=\infty$. Нулевое рекуррентное, возвратное, null-recurrent
\item $f_i<1$, $\E(T_i)<\infty$. Положительное рекуррентное, возвратное, positive recurrent
\end{enumerate}


Определение. Марковская цепь называется несократимой, irreducible, если из любого состояния можно с положительной вероятностью попасть в любое за некоторое количество ходов.

Определение. Регулярная цепь, regular. Существует число $k$, такое, что за $k$ шагов можно с положительной вероятностью из любого состояния попасть в любое.

Утверждение. Регулярная --- частный случай несократимой. 

Упражнение. Приведите пример несократимой, нерегулярной.



Определение. Вектор $\pi$ называется стационарным вектором марковской цепи, если $\pi=\pi P$.

Упражнение. Приведите пример цепи с несколькими стационарными векторами. Без стационарных векторов.


\begin{theorem}
У несократимой марковской цепи стационарный вектор существует если и только если все состояния являются положительными возвратными. В этом случае он единственный и $\pi_j=1/E(T_j)$.
\end{theorem}



Лемма 1. Если в несократимой Марковской цепи существует стационарный вектор, то все состояния являются положительными возвратными и $\pi_j=1/\E(T_j)$.


Лемма 2. Если хотя бы одно состояние несократимой цепи является положительным возвратным, то хотя бы один стационарный вектор существует.


\begin{theorem}
В несократимой непериодичной цепи, где все состояния положительные возвратные, стационарный вектор является предельным для любого стартового распределения. Иными словами, для любого вектора вероятностей $p$: $\lim p P^{n}=\pi$.
\end{theorem}

Пример. Несократимая цепь, где все состояния положительные возвратные. Есть стационарный вектор. Нет сходимости к нему.


Примеры для бесконечного числа состояний. Несократимая апериодичная без стационарного вектора. Случайное блуждание с сильным сносом.


Доказательства. Ross, Second course in Stochastic Processes, p.149-157

Для конечных всё несколько проще. 

\begin{theorem}
В цепи с конечным числом состояний регулярность равносильна апериодичности и несократимости.
\end{theorem}

? proof


Теорема Брауэра.


Применения теоремы Брауэра.

Уронили карту мира. Можно взять иголку и проткнуть карту так, чтобы иголка и на карте и на земле воткнулась в одну точку.


Лемма о причесывании ежа. Ежа нельзя причесать. Одна иголка будет стоять вертикально.


На марсе есть тайфуны. 


Как не мешай кофе есть неподвижная точка.


Есть равновесие Нэша.


На земном шаре есть две антиподальные точки с равной температурой.


\begin{theorem}
В цепи с конечным числом состояний есть стационарный вектор.
\end{theorem}


