\section{Что такое байесовская сеть?}

Здесь картинка.

На картинке:
\begin{enumerate}
\item Кружочками обозначаются случайные величины.
\item Стрелочками --- причинно следственные связи:

\begin{tikzpicture}[grow=right]
\tikzstyle{mycircle} = [circle, draw, minimum width=16pt, inner sep=0pt] % node style
\tikzstyle{edge from parent}=
     [-angle 45,draw, % рисуем стрелочку
     edge from parent path={(\tikzparentnode) -- (\tikzchildnode)}] % это магическое заклинание направляющее ребра к центру узла, а не к точке под узлом:     
\tikzstyle{every node}=[mycircle]
\node {$X$} % создаем узел 
    child { node {$Y$} } ;
\end{tikzpicture}

Значение величины $X$ становится известно раньше значиня $Y$. Закон распределения величины $Y$ зависит от значения величины $X$.
\end{enumerate}

Несколько терминов:
\begin{enumerate}
\item Вилка (fork)
\item Коллайдер, перевернутая вилка (collider, inverted fork)
\item Путь\footnote{В теории графов термином путь называют то, что мы называем направленным путем} (trail, path) от $A$ до $B$ --- последовательность вершин от вершины $A$ до вершины $B$, в которой переходы могут делаться и по стрелочкам и против стрелочек
\item Направленный путь от $A$ до $B$ --- последовательность вершин от вершины $A$ до вершины $B$, в которой переходы делаются только по стрелочкам
\item Потомок. Узел $Y$ называется потомком узла $X$, если существует направленный путь от $X$ до $Y$.
\item Предок. Узел $X$ называют предком узла $Y$, если существует направленный путь от $X$ до $Y$.
\item Прямой родитель. Узел $X$ --- прямой родитель узла $Y$, если $X\to Y$.

\end{enumerate}


По байесовской сети легко определить зависимость и условную зависимость величин. Сначала разберемся с зависимостью.

Определение. Направленный граф без циклов $G$ называется байесовской сетью случайных величин $X_1$, \ldots, $X_n$, если для любого узла $X$ выполнено условие:
\begin{equation}
X \perp \mbox{non descendant}(X) \mid \mbox{parents}(X)
\end{equation}


Теорема. Величины $X$ и $Y$ независимы, если выполнены все три условия 
\begin{enumerate}
\item Нет направленного пути от $X$ до $Y$
\item Нет направленного пути от $Y$ до $X$
\item Не существует такой величины $Z$, от которой был бы направленный путь и до $X$ и до $Y$
\end{enumerate}


Упражнение. Найдите все пары независимых величин.
\todo[inline]{Нарисовать какую-нибудь картинку}

\section{Условная независимость}

Условная независимость. События $A$ и $B$ называются условно независимыми при условии, что событие $C$ произошло, если $\P(AB \mid C)=\P(A \mid C)\P(B \mid C)$

Примеры.
\begin{enumerate}
\item Независимые, но условно зависимые события.
\item Зависимые, но условно независимые события.
\item Независимы при условии $C$, зависимы при отрицании $C$
\end{enumerate}

Дискретные случайные величины $X$ и $Y$ условно независимы при условии $Z$, если для любых $x$, $y$ и $z$:
\begin{equation}
\P(X=x,Y=y \mid Z=z) = \P(X=x \mid Z=z)\cdot \P(Y=y \mid Z=z)
\end{equation}

Условную независимость величин обозначают $X \perp Y \mid Z$


Примечание: некоторые авторы пишут $A \perp B \mid C$ для событий, под этой записью подразумевается на самом деле сразу два условия:

\begin{equation}
A \perp B \mid C \Leftrightarrow 
\begin{cases}
A \mbox{ и } B \mbox{ независимы при условии } C \\
A \mbox{ и } B \mbox{ независимы при условии } C^c \\
\end{cases}
\end{equation}


Определение. Путь между $X$ и $Y$ называют $d$-разделенным (d-separated, directionally separated) множеством узлов $Z$ если выполнено хотя бы одно из условий
\begin{enumerate}
\item узел из $Z$ разрывает последовательное соединение на пути
\item узел из $Z$ разрывает <<вилку>> на пути
\item на пути есть <<коллайдер>>, не являющийся узлом из $Z$ и не содержащий узел из $Z$ в качестве одного из потомков
\end{enumerate}


Можно эквивалентно говорить о том, что путь между $X$ и $Y$ НЕ является $d$-разделенным узлом $Z$, если выполнены оба условия:
\begin{enumerate}
\item любой коллайдер на пути либо сам является узлом из множества $Z$, либо имеет потомка из множества $Z$
\item никакой другой узел на пути не входит в множество $Z$
\end{enumerate}


Случайные величины $X$ и $Y$ условно независимы при условии $Z$, если узлы $X$ и $Y$ являются $d$-разделенными узлом $Z$.


Упражнения
\begin{tikzpicture}[grow=up]
\tikzstyle{mycircle} = [circle, draw, minimum width=16pt, inner sep=0pt] % node style
\tikzstyle{edge from parent}=
     [angle 45-,draw, % рисуем стрелочку
     edge from parent path={(\tikzparentnode) -- (\tikzchildnode)}] % это магическое заклинание направляющее ребра к центру узла, а не к точке под узлом:     
\tikzstyle{every node}=[mycircle]
\node {$X_8$} % создаем узел 
    child { node {$X_6$}
	    child { node {$X_4$}        
		    child { node {$X_1$} } }
	    child { node {$X_5$}        
		    child { node {$X_2$} }
		    child { node {$X_3$} } } }
    child { node {$X_7$} } ;
\end{tikzpicture}

Проверьте независимость $X_1 \perp X_2$, $X_1 \perp X_2 \mid X_8$, $X_1 \perp X_2 \mid X_7$, $X_1 \perp X_2 \mid X_6$

