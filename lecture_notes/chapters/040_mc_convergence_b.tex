\section{Сходимость марковских цепей}

Введем обозначения:

\begin{enumerate}
\item $T_i$ --- время до посещения состояния $i$, если начать в состоянии $i$
\end{enumerate}

Случайная величина $T_i$ особенна тем, что она может принимать нечисловое значение $+\infty$, если цепь начала в состоянии $i$, но никогда в него не вернулась. 

Для подсчета математического ожидания $\E(T_i)$ мы примем соглашение, что $0\cdot\infty=0$. Другими словами, если $\P(T_i=\infty)=0$, то $\E(T_i)=\sum k\P(T_i=k)$, а если $\P(T_i=\infty)>0$, то $\E(T_i)=\infty$.

Вероятность того, что цепь из состояния $i$ когда-либо вернется в состояние $i$ мы обозначим $f_i$, то есть $f_i=1-\P(T_i=\infty)$.

Классификация состояний:
\begin{enumerate}
\item $f_i=1$, $\E(T_i)<\infty$. Транзиентное, переходное, transient
\item $f_i<1$, $\E(T_i)=\infty$. Нулевое рекуррентное, возвратное, null-recurrent
\item $f_i<1$, $\E(T_i)<\infty$. Положительное рекуррентное, возвратное, positive recurrent
\end{enumerate}


Утв. Если $a$ --- рекуррентное состояние и из $a$ можно попасть в $b$, то $b$ --- рекуррентное состояние и из $b$ можно попасть в $a$.



По ходу прикольное упражнение: Вася хочет переплюнуть свой собственный вчерашний прыжок. Сколько дней ему потребуется для этого?

Определение. Марковская цепь называется несократимой, irreducible, если из любого состояния можно с положительной вероятностью попасть в любое за некоторое количество ходов.



Утверждение. В несократимой цепи либо все состояние рекуррентные, либо все --- транзиентные. 

Примеры. (три точки), (случайное блуждание), (сократимая)


Определение. Регулярная цепь, regular. Существует число $k$, такое, что за $k$ шагов можно с положительной вероятностью из любого состояния попасть в любое.

Утверждение. Регулярная --- частный случай несократимой. 

Упражнение. Приведите пример несократимой, нерегулярной.



Определение. Вектор $\pi$ называется стационарным вектором марковской цепи, если $\pi=\pi P$.

Упражнение. Приведите пример цепи с несколькими стационарными векторами. Без стационарных векторов.




Период состояния: число $k$ называется периодом состояния $a$ если 

$k=gcd(n\mid P^n_{aa}>0)$. 


Если из состояния $a$ в $a$ можно добраться за $n$ ходов, то $n$ делится на $k$.


Состояние с периодом равным 1 называется апериодичным. 


Пример. (треугольник), (случайное блуждание)



Теорема. Если цепь конечная, апериодичная, несократимая, то она регулярная.

Приведите пример бесконечной несократимой, апериодичной нерегулярной цепи.


Утверждение. Если цепь конечна и несократима, то каждое состояние является положительным рекуррентным.

Пример. (случайное блуждание)


Утверждение. Если цепь конечна и несократима, и $T_{i\to j}$ --- время перехода из $i$ в $j$, то $\P(T_{i\to j}=\infty)=0$.  


Опредение. Стационарный вектор. 


Теорема. В регулярной цепи существует стационарный вектор.

%%%

$\pi P=\pi$, значит и $\pi P^n=\pi$.



\begin{equation}
\sum_{\pi_j>0} \pi_j=
\end{equation}


Утверждение. Левые и правые собственные числа матрицы $P$ совпадают.

Утверждение. $P\cdot (1,1,\ldots,1)'=(1,1,\ldots,1)'$, т.е. есть собственное число 1.


Теорема (Дёблин, Винни-Пух и Пятачок)

Винни-пУх --- по оси Y

Пятачок (Хряк) --- по оси X

Если $\pi$ --- стационарный вектор марковской цепи, то $uP^n\to \pi$


Цепь $P^*$ определяем по принципу:

$P^*\left((a_1,a_2)\to (b_1,b_2)\right)=P(a_1\to b_1)P(a_2\to b_2)$

$T$ --- время достижения состояния вида $(s,s)$


$\P(T=\infty)=0$, поэтому $\lim_{n\to\infty} P(T>n)=0$

Винни-Пух и Пятачок обязательно встретятся!

Как только они встретятся условный прогноз их  будущего одинаковый:

$\P(X_n=a\mid n\geq T)=\P(Y_n=a\mid n\geq T)$

$\P(X_n=a \cap n\geq T)=\P(Y_n=a \cap n\geq T)$


$\P(X_n=a)-\P(X_n=a \cap n<T)=\P(Y_n=a)-\P(Y_n=a \cap n> T)$




Упражнение. Есть марковская цепь $P$, есть вектор вероятностей $\pi$. Строим марковскую цепь $Q$ по принципу $Q_{i\to j}=\min \{ P_{ij}, P_{ji}\frac{\pi_j}{\pi_i} \}$ и $Q_{ii}=1-\sum_{j\neq i} Q_{ij}$. Найдите её стационарное состояние!


(проделать для случая 3 состояний)

Это был Алгоритм Метрополиса-Хастингса.




\begin{theorem}
У несократимой марковской цепи стационарный вектор существует если и только если все состояния являются положительными возвратными. В этом случае он единственный и $\pi_j=1/E(T_j)$.
\end{theorem}



Лемма 1. Если в несократимой Марковской цепи существует стационарный вектор, то все состояния являются положительными возвратными и $\pi_j=1/\E(T_j)$.


Лемма 2. Если хотя бы одно состояние несократимой цепи является положительным возвратным, то хотя бы один стационарный вектор существует.


\begin{theorem}
В несократимой непериодичной цепи, где все состояния положительные возвратные, стационарный вектор является предельным для любого стартового распределения. Иными словами, для любого вектора вероятностей $p$: $\lim p P^{n}=\pi$.
\end{theorem}

Пример. Несократимая цепь, где все состояния положительные возвратные. Есть стационарный вектор. Нет сходимости к нему.


Примеры для бесконечного числа состояний. Несократимая апериодичная без стационарного вектора. Случайное блуждание с сильным сносом.


Доказательства. Ross, Second course in Stochastic Processes, p.149-157

Для конечных всё несколько проще. 

\begin{theorem}
В цепи с конечным числом состояний регулярность равносильна апериодичности и несократимости.
\end{theorem}

? proof


Теорема Брауэра.


Применения теоремы Брауэра.

Уронили карту мира. Можно взять иголку и проткнуть карту так, чтобы иголка и на карте и на земле воткнулась в одну точку.


Лемма о причесывании ежа. Ежа нельзя причесать. Одна иголка будет стоять вертикально.


На марсе есть тайфуны. 


Как не мешай кофе есть неподвижная точка.


Есть равновесие Нэша.


На земном шаре есть две антиподальные точки с равной температурой.\\


Теорема. В Марковской цепи с конечным числом состояний всегда есть стационарный вектор.\\
Идея доказательства --- через теорему Брауэра.\\
Теорема Брауэра. Если $B^n$ --- шар в n-мерном пространстве $(x\in R\mid x_1^2+x_2^2+x_3^2+...+x_n^2<=1)$, и $f$ --- непрерывная функция $f:B^n\to B^n$, то $\exists x : f(x)=x$, то есть найдется точка, которая отобразится сама в себя.\\
На примерах:
\begin{itemize}
\item Если мы проводим с кофе непрерывное преобразования путем помешивания, то, независимо от того, как мешать кофе, будет частичка кофе, которая останется после помешивания на своем месте. 
\item Если бросим карту Москвы где-нибудь на территории Москвы, будет точка, которая идеально совпадет с точкой на местности.
\end{itemize} 
<<Показательство>>от противного.\\
У нас есть круг. Через каждую точку $x$ мы проводим луч $f(x)$ до края круга. Тогда появляются точки $g(x)$ на окружности. Получается, что $g(x)$ --- непрерывное преобразование круга в окружность. При этом точки, которые изначально были на окружности, переведутся сами в себя. Непрерывность преобразования $g(x)$ наглядно: растягиваем постепенно вокруг точки $x$ резиночку, и на такое же малое расстояние растягиваем резиночку вокруг $g(x)$. Если растянуть резиночку до размеров окружности, то резиночка $x$ совпадет с окружностью, а резиночка $g(x)$ обернется вокруг окружности уже несколько раз $\Rightarrow$  при непрерывности $g(x)$ допущение об отсутствии неподвижных точек неверно.\\
Следствия из теоремы Брауэра:
\begin{itemize}
\item Лемма о причесывании ежа.\\
Круглого ежа без пробора, у которого иголки непрерывно меняют направление(несильно), нельзя причесать: хотя бы одна иголка останется стоять вертикально,
\item Существование тайфунов
\item Существуют две диаметрально противоположные точки, у которых одинаковая температура и влажность.\\
В каждой точке смотрим на 2 показателя --- температуру  ($t$) и влажность ($\rho$) --- и строим вектор $t_{$здесь$}-t{$напротив$};\rho_{$здесь$}-\rho_{$напротив$}$. Обязательно найдется точка, в которой этот вектор будет нулевым.
\end{itemize}



%%%%






