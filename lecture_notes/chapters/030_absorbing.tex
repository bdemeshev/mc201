\section{Поглощающие состояния}
Определение. Состояние $i$ называется поглощающим(absorbing), если $P_i=1$.\\
Определение. Состояние $i$ называется переходным, если $P_i<1$.\\
Определение. Вся цепь Маркова называется поглощающей, если из любого состояния можно попасть в какое-нибудь поглощающее, необязательно за один ход.\\
Матрицу поглощающей цепи всегда можно представить в виде:
$P=
\left(\begin{array}{cc}
Q & R\\
0 & I\\
\end{array}\right)$, где $Q$ --- квадратная матрица, $R$ --- прямоугольная матрица $I$ --- единичная матрица, $0$ --- нулевая.\\
Теорема. Вне зависимости от стартового узла(распределения), поглощающая Марковская цепь достигает поглощающего состояния с вероятностью $1$.\\
Доказательство. Из любого узла есть хотя бы одна траектория, ведущая к поглощающему узлу. Все эти траектории ограничены общей максимальной длиной $m$. Жизнь цепи поделена на несколько участков по $m$ шагов. За первые $m$ шагов есть вероятность $p>0$ завершиться, поскольку из любого состояния есть траектория до поглощающего узла. Вероятность того, что цепь не достигнет поглощающего узла за $m$ шагов --- $(1-p)<1$.\\
 $p("никогда не достигнем поглощающего узла")\le (1-p)(1-p)(1-p)...=0\\
$
Другая формулировка: $lim_{n\to\infty} Q^n=0\\
$
Обоснование:\\
$\P_{ij}=\P_{i \to j}$ --- вероятность попасть из состояния $i$ в  $j$\\
$\P^2_{ij}=\P^2_{i \to j}$ --- вероятность попасть из состояния $i$ в  $j$ за два хода.\\
$P \cdot P=
\left(\begin{array}{cc}
Q & R\\
0 & I
\end{array}\right) \cdot
\left(\begin{array}{cc}
Q & R\\
0 & I
\end{array}\right)=
\left(\begin{array}{cc}
Q^2 &$ Гадость$\\
0 & I \end{array}\right)$\\
Пример с Винни-Пухом.
Винни-Пух хромает на левую ногу и ходит с вероятностью $3/4$ направо, $1/4$ --- налево. Он живет с друзьями в своем мире:
\includegraphics[width=80mm]{chapters/Vinnie.png}
Нора Кролика и домик Винни-Пуха --- поглощающие состояния для Винни. Нам нужно найти ожидаемое количество раз, которое Винни-Пух проведет у Пятачка. Пусть Винни-Пух стартует от дерева с пчелами, и это состояние $1$. Дерево --- состояние $2$, дом Пятачка --- состояние $3$, дом Винни-Пуха --- состояние $4$, и нора Кролика --- $5$. Тогда матрица перехода:\\
$\left(\begin{array}{ccccc}
0 & 3/4 & 0 & 0 & 1/4\\
1/4 & 0 & 3/4 & 0 & 0\\
0 & 1/4 & 0 & 3/4 & 0\\
0 & 0 & 0 & 1 & 0\\
0 & 0 & 0 & 0 & 1\\
\end{array}\right)$\\
Пусть $Y$ --- время(в количествах посещений), которое Винни-Пух проведет у Пятачка до поглощения своим домом или норой Кролика.
$Y=Y_0+Y_1+Y_2+Y_3+...$, где $Y_i=
\begin{cases}
  1,$если в момент времени t=i Винни-Пух у Пятачка$ \\
  0,$если в момент времени t=i Винни-Пух не у Пятачка$
\end{cases}\\
E(Y)=E(Y_0)+E(Y_1)+E(Y_2)+...=p(Y_0=1)+p(Y_1=1)+p(Y_2=1)+...=P^0_{13}+P^1_{13}+P^2_{13}+...=Q^0_{13}+Q^1_{13}+Q^2){13}+...\Rightarrow$
Представляет интерес матрица $N=Q^0(=I)+Q^1+Q^2+Q^3+...$, которая будет показывать ожидаемое количество посещений всех узлов при разных стартовых точках.
Как сумма бесконечно убывающей геометрической прогрессии, $N=(I-Q)^{-1}$.\\
Докажем это. 
\begin{enumerate}
\item Матрица $(I-Q)$ обратима. Пойдем от противного: предположим, что 
$\exists x| x(I-Q)=0\\
x=xQ\\
xQ=xQ^2
\Rightarrow x=xQ^2=xQ^3=...\Rightarrow x=xQ^n\\
\lim_{n\to\infty}x=lim_{n\to\infty}xQ^n$\\
Единственное решение такой системы --- нулевое.
\item $(I+Q^1+Q^2+...+Q^n)(I-Q)=I-Q^{n+1}\\
\lim_{n\to\infty}[(I+Q^1+Q^2+...+Q^n)(I-Q)=\lim_{n\to\infty}(I-Q^{n+1})\\
(I+Q^1+Q^2+...+Q^n)(I-Q)=I (\lim_{n\to\infty}Q^{n+1}=0)\\
I+Q^1+Q^2+...+Q^n=(I-Q)^{-1}$ (можем так делать, поскольку матрица обратима).\\
\end{enumerate}
Найдем $(I-Q)^{-1}$ для Винни-Пуха:\\
$(I-Q)=
\left(\begin{array}{ccc}
1 & -3/4 & 0\\
-1/4 & 1 & -3/4\\
0 & -1/4 & 1
\end{array}\right)\\
(I-Q)^{-1}=
\left(\begin{array}{ccc}
1,3 & 1,2 & 0,9\\
0,4 & 1,6 & 1,2\\
0,1 & 0,4 & 1,3
\end{array}\right)\\
$ В этой матрице $N_{ij}$ --- среднее число посещений состояния $j$ при старте из состояния $i$, то есть среднее число визитов к Пятачку при старте с дерева с пчелами --- $N_{13}=0,9$. Из этой матрицы легко найти и ожидаемое время до поглощения при старте с какого-либо состояния: к примеру, если стартовать с дерева с пчелами(узел 1), то пройдет: $1,3+1,2+0,9=3,4$.\\

Задача про Дашу и Глашу.\\
Вспомним условие: Даша и Глаша подкидывают <<правильную>> монетку до тех пор, пока не выпадет ОРО или РРР. Глаша выигрывает в случае РРР, Даша ---  в случае ОРО.\\
Построим дерево игры:
\begin{tikzpicture}[->,>=stealth',shorten >=1pt,auto,node distance=1.5cm,
  thin,main node/.style={circle, draw, minimum width=16pt, inner sep=0pt}]
\node[main node] (1) {START};
\node[main node] (2) [above right of=1] {P};
\node[main node] (3) [right of=2] {PP};
\node[main node] (4) [below right of=1] {O};
\node[main node] (5) [right of=4] {OP};
\node[main node] (6) [right of=3] {PPP};
\node[main node] (7) [right of=5] {OPO};
\path
(1) edge node {} (2)
	edge node {} (4)
(2) edge node {} (3)
	edge node {} (4)
(3) edge node {} (4)
	edge node {} (6)
(4) edge node {} (5)
(5) edge node {} (7)
(6) edge [loop right] node {} (6)
(7) edge [loop right] node {} (7);
\end{tikzpicture}\\
Построим матрицу перехода для этой игры:\\
$\left(\begin{array}{ccccccc}
0 & 1/2 & 0 & 1/2 & 0 & 0 & 0\\
0 & 0 & 1/2 & 1/2 & 0 & 0 & 0\\
0 & 0 & 0 & 1/2 & 0 & 1/2 & 0\\
0 & 0 & 0 & 1/2 & 1/2 & 0 & 0\\
0 & 0 & 0 & 1/2 & 0 & 0 & 1/2\\ 
0 & 0 & 0 & 0 & 0 & 1 & 0\\
0 & 0 & 0 & 0 & 0 & 0 & 1\\
\end{array}\right)$\\
Сначала решим руками. Пусть $g_i$ --- вероятность того, что выиграет Глаша (выпадет комбинация <<РРР>>), если стартовать с узла $i$. Тогда:\\
$\left\{$
$g_5=1/2g_3+1/2\cdot0=1/2g_3\\
g_4=1/2g_5+1/2g_4=1/2g_3\\
g_2=1/2g_4+1/2g_3=1/4g_3\\
g_1=1/2g_2+1/2g_4=5/8g_3\\
g_3=1/2\cdot1+1/2g_4
\Rightarrow g_1=5/12\\
$
Тот же результат можно было найти обобщенно: как матрицу $B$, элементы которой --- вероятности попасть в поглощающие состояния, стартуя с разных узлов. Чтобы было удобнее работать с матрицей $P=
\left(\begin{array}{cc}
Q & R\\
0 & I\\
\end{array}\right)$, занумеруем по-разному поглощающие и переходные состояния: $i$ --- номер переходного состояния, $j$ --- номер поглощающего состояния. Тогда\\
$B_{i\to j}=R_{ij}+[Q_{i\to 1}B_{1 \to j}+Q_{i\to 2}B_{2\to j}+...]\\
B_{i\to j}=R_{ij}+\sum_{k=1}^{n}Q_{i\to k}B_{k\to j}\\
B_{i\to j}=R_{ij}+(QB)_{ij}\\
B=R+QB\\
(I-Q)B=R\\
B=(I-Q)^{-1}R=NR\\
\lim_{n\to \infty}P^n=\left(\begin{array}{cc}
0 & B\\
0 & I\\
\end{array}\right)$\\
Для Винни-Пуха:\\
$B=NR=
\left(\begin{array}{ccc}
1,3 & 1,2 & 0,9\\
0,4 & 1,6 & 1,2\\
0,1 & 0,4 & 1,3
\end{array}\right)\cdot
\left(\begin{array}{cc}
0 & 1/4\\
0 & 0\\
3/4 & 0
\end{array}\right)=
\left(\begin{array}{cc}
0,675 & 0,325\\
0,9 & 0,1\\
0,975 & 0,025\\ 
\end{array}\right)$\\
В этой матрице мы можем, к примеру посмотреть вероятность попадания во 2 поглощающее состояние из 2 переходного, то есть вероятность того, что Винни-Пух навсегда останется у Кролика, если стартует с дерева без пчел.




















