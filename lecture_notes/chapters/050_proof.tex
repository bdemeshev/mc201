

Пусть $\pi$ --- собственный вектор. Если в $\pi$ нет элементов разного знака, то мы поделим вектор $\pi$ на $\sum \pi_i$ и получим собственный вектор из вероятностей. Трудность возникает только в том случае, если в векторе $\pi$ есть элементы разного знака.

Рассмотрим этот потенциальный случай подробнее. Для удобства обозначений введем матрицы $I_{+}$, $I_{-}$ и вектор столбец $u$.
В матрице $I_{+}$ на месте $ii$ стоит $1$ если $\pi_i>0$, во всех остальных местах матрицы $I_{+}$ стоят нули. В матрице $I_{+}$ на месте $ii$ стоит $1$ если $\pi_i\leq 0$, во всех остальных местах матрицы $I_{-}$ стоят нули. Заметим, что $I_{+}+I_{-}=I$.

Вектор столбец $u$ состоит только из единиц, $u=(1,1,\ldots,1)'$.


Вектор $u$ --- правый собственный вектор матрицы $P$ с собственным числом $1$. Замечаем, что $P^n u=u$ и $0<P_n I_{+} u<u$. 

Вектор $\pi$ --- левый собственный вектор матрицы $P$ с собственным числом $1$, $\pi = \pi P^n$.

Домножим справа на $I_{+}u$:

\begin{equation}
\pi I_{+} u = \pi P^n I_{+} u
\end{equation}

Бесплатно вставим умножение на единичную матрицу, $I=I_{+}+I_{-}$.
\begin{equation}
\pi I_{+} u = \pi (I_{+}+I_{-})P^n I_{+} u
\end{equation}

Раскроем скобки и перенесём $\pi I_{+}P^n I_{+}u$ в левую часть:
\begin{equation}
\pi I_{+} u - \pi I_{+}P^n I_{+} u = \pi I_{-} P^n I_{+} u
\end{equation}

Вынесем общий множитель слева:
\begin{equation}
\pi I_{+}(I - P^n I_{+}) u = \pi I_{-} P^n I_{+} u
\end{equation}

Обратим внимание, что:

\begin{enumerate}
\item вектор-столбец $P^n I_{+} u$ состоит из строго положительных чисел
\item вектор-столбец $(I-P^n I_{+}) u$ состоит из строго положительных чисел
\item вектор-строка $\pi I_{+}$ состоит из неотрицательных чисел
\item вектор-строка $\pi I_{-}$ состоит из неположительных чисел
\end{enumerate}

Это возможно только тогда, когда все числа в векторе $\pi$ одного знака.



