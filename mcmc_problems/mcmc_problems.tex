\documentclass[pdftex,12pt,a4paper]{article}

\input{/home/boris/science/tex_general/title_bor_utf8}

\title{Задачи к курсу mcmc}
\author{Борис Демешев}
\date{\today}

% временное решение
%\newcommand{\solution}[1]{ {\tiny #1} }
\newcommand{\solution}[1]{}

\newcommand{\problem}[1]{#1}




\begin{document}
\parindent=0 pt % отступ равен 0

\section{Байесовские сети}


\section{Марковские цепи}
\begin{enumerate}
\item Шахматный конь начинает в клетке A1. Каждый свой ход он выбирает равновероятно из возможных. Какова вероятность того, что через много-много ходов он окажется в клетке H8? Сколько в среднем длится путь от клетки A1 до клетки A1?
\end{enumerate}
\solution{}

\section{Простые байесовские задачи}
\begin{enumerate}
\item Случайные величины $X_i$ независимы и одинаково распределены с табличкой

\begin{tabular}{c|ccc}
$X$ & 1 & 2 & 6 \\ 
\hline 
$\P()$ & $\beta$ & $2\beta$ & $1-3\beta$ \\ 
\end{tabular} 

Известно, что $X_1=1$, $X_2=2$, $X_3=2$, $X_4=4$. 
\begin{enumerate}
\item Найдите оценку $\hat{\beta}$ методом моментов
\item Найдите оценку $\hat{\beta}$ методом максимального правдоподобия
\item Предположим, что $\beta$ равномерно на отрезке $[0;1/3]$. Найдите апостериорную условную функцию плотности $\beta$ с учётом полученных наблюдений. С какой функцией она совпадает?
\item Предположим, что $\beta$ имеет функцию плотности $f(t)=18t$ на отрезке $[0;1/3]$. Найдите апостериорную функцию плотности $\beta$. 
\end{enumerate}



\end{enumerate}

\section{Компьютерные}

\begin{enumerate}
\item Макар-Лиманов
\item Своя собственная регрессия 
\end{enumerate}



\end{document}