\documentclass[pdftex,12pt,a4paper]{article}

\input{/home/boris/science/tex_general/title_bor_utf8}

\title{Задачи к курсу mcmc}
\author{Борис Демешев}
\date{\today}

% временное решение
%\newcommand{\solution}[1]{ {\tiny #1} }
\newcommand{\solution}[1]{}

\newcommand{\problem}[1]{#1}




\begin{document}
\parindent=0 pt % отступ равен 0

\section{Байесовские сети}


\section{Марковские цепи}
\begin{enumerate}
\item Шахматный конь начинает в клетке A1. Каждый свой ход он выбирает равновероятно из возможных. Какова вероятность того, что через много-много ходов он окажется в клетке H8? Сколько в среднем длится путь от клетки A1 до клетки A1?
\end{enumerate}
\solution{}

\section{Простые байесовские задачи}
\begin{enumerate}
\item Случайные величины $X_i$ независимы и одинаково распределены с табличкой

\begin{tabular}{c|ccc}
$X$ & 1 & 2 & 6 \\ 
\hline 
$\P()$ & $\beta$ & $2\beta$ & $1-3\beta$ \\ 
\end{tabular} 

Известно, что $X_1=1$, $X_2=2$, $X_3=2$, $X_4=4$. 
\begin{enumerate}
\item Найдите оценку $\hat{\beta}$ методом моментов
\item Найдите оценку $\hat{\beta}$ методом максимального правдоподобия
\item Предположим, что $\beta$ равномерно на отрезке $[0;1/3]$. Найдите апостериорную условную функцию плотности $\beta$ с учётом полученных наблюдений. С какой функцией она совпадает?
\item Предположим, что $\beta$ имеет функцию плотности $f(t)=18t$ на отрезке $[0;1/3]$. Найдите апостериорную функцию плотности $\beta$. 
\end{enumerate}



\end{enumerate}

\section{Компьютерные}

\begin{enumerate}
\item Макар-Лиманов
\item Своя собственная регрессия 


\item Используя алгоритм Метрополиса-Хастингса сгенерите выборку для биномиального распределения $Bin(n,p)$ из равновероятного на множестве $\{0,1,2,\ldots,n\}$

\item Используя алгоритм Метрополиса-Хастингса сгенерите выборку для биномиального распределения $Bin(n,p)$ из симметричного случайного блуждания на $\ZZ$

\item Используя алгоритм Метрополиса-Хастингса сгенерите выборку для геометрического распределения $Geom(p)$ из симметричного случайного блуждания на $\ZZ$

\item Используя алгоритм Метрополиса-Хастингса сгенерите выборку для пуассоновского распределения $Pois(\lambda)$ из симметричного случайного блуждания на $\ZZ$

\item Используя алгоритм Метрополиса-Хастингса сгенерите выборку для функции плотности $\pi(x)\sim \exp(-x^2)(3+x^2+\cos x)$ из нормального $N(0,1)$. Из нормального $N(0,\sigma^2)$

\item Используя алгоритм Метрополиса-Хастингса сгенерите выборку для функции плотности $\pi(x)\sim \exp(-x^2)(3+x^2+\cos x)$ из случайного блуждания $X_{t+1}=X_t+\varepsilon_t$, где $\varepsilon_t\sim N(0,1)$. Вариант с $N(0,\sigma^2)$

\item Используя алгоритм Метрополиса-Хастингса сгенерите выборку для стандартного нормального распределения $N(0,1)$ из случайного блуждания $X_{t+1}=X_t+\varepsilon_t$, где $\varepsilon_t\sim U[-1,1]$

\item Используя алгоритм Метрополиса-Хастингса сгенерите выборку для двумерного нормального распределения $N(0,A)$, $A=\left(\begin{array}{cc}
4 & -1 \\ 
-1 & 2
\end{array}\right)$  
из случайного блуждания 
$X_{t+1,i}=X_{t,i}+\varepsilon_{t,i}$, 
где $\varepsilon_{t,i}\sim U[-1,1]$.

\item Используя алгоритм Метрополиса-Хастингса сгенерите выборку для двумерного распределения с функцией плотности 
$p(x,y)=\exp(-4x^2-6y^2+2x-y+xy)$, $x>0$, $y>0$ из случайного блуждания 
$X_{t+1,i}=X_{t,i}+\varepsilon_{t,i}$, 
где $\varepsilon_{t,i}\sim U[-1,1]$.



\end{enumerate}




\end{document}